%------------------------------------------------------------------------------
\chapter{\ix{LaTeX}}
\label{sec:share000.latex}
%------------------------------------------------------------------------------
\section{Introduction}
%------------------------------------------------------------------------------

From WikiPedia:
\begin{quote}
LaTeX is a document preparation system and document markup language. It is
widely used for the communication and publication of scientific documents in
many fields, including mathematics, physics, and computer science.
\end{quote}

\noindent
LaTeX was chosen because the document markup format is fully text based. This
allows archiving, comparing and merging of LaTeX sources in an identical way as
C++ programming code. Such an approach enables the usage of tools and
mechanisms that encourage or enforce a tight coupling between source code and
its documentation.

\noindent
From WikiPedia:
\begin{quote}
LaTeX follows the design philosophy of separating presentation from content, so
that authors can focus on the content of what they are writing without
attending simultaneously to its visual appearance. In preparing a LaTeX
document, the author specifies the logical structure using simple, familiar
concepts such as chapter, section, table, figure, etc., and lets the LaTeX
system worry about the formatting and layout of these structures. It therefore
encourages the separation of layout from content while still allowing manual
typesetting adjustments where needed. This concept is similar to the mechanism
by which many word processors allow styles to be defined globally for an entire
document or the use of Cascading Style Sheets to style HTML. The LaTeX system
is a markup language that also handles typesetting and rendering.
\end{quote}

\noindent
In this project we also aim to benefit that "authors can focus on the content
of what they are writing without attending simultaneously to its visual
appearance". The separation of presentation from content also prevents
typographical differences between different developers. So eventually the
contributions of different participants will integrate seamlessly. 

A very good integral reference is \cite{LaTeX}. It counts 706 pages, just to
indicate that the scope is rather broad. The scope of this chapter is limited
to the mark-up commands a developer may need to document his component. All
mark-up commands listed here are at least already used once in this project,
so besides this chapter and \cite{LaTeX} a real-life example can be found in
the archive. This allows us to keep the explanation of these mark-up commands
very short.

%------------------------------------------------------------------------------
\section{Headings}\label{sec:LaTeX.Headings}
%------------------------------------------------------------------------------

The (component) developer is typically only confronted with sectioning a single
chapter. In this project we put the content of a single chapter in a single
file, typically called "chapter.tex". Such a file will typically start with

\index{chapter@\textbackslash chapter\{\}}
\index{section@\textbackslash section\{\}}
\begin{lstlisting}[language=]
%------------------------------------------------------------------------------
\chapter{Title of the chapter}
%------------------------------------------------------------------------------
\section{Title of the first section in this chapter}
%------------------------------------------------------------------------------
\end{lstlisting}

\noindent
The !%---...! is a project convention. Lines starting with !%! are comments,
and do not end up in the actual document. This convention is a visual aid to
show the sectioning structure of you chapter in your source.

It is also possible to have sub-sections in a section. Such a sub-sections will
not end up in the table of contents. Please do not use !\subsubsection!,
!paragraph! and !subparagraph!. A sub section with the commenting convention
looks as follows:

\index{subsection@\textbackslash subsection\{\}}
\begin{lstlisting}[language=]
%..............................................................................
\subsection{Title of a sub-section}
\end{lstlisting}

%------------------------------------------------------------------------------
\section{Enumerating}
%------------------------------------------------------------------------------

LaTeX supports three forms of enumeration:
\begin{description*}
\item[description] This enumeration itself is of description type
   \index{begin{description}1@\textbackslash begin\{description\}}
   \index{begin{description}2@\textbackslash begin\{description*\}}
   \index{end{description}1@\textbackslash end\{description\}}
   \index{end{description}2@\textbackslash end\{description*\}}
   \index{item1@\textbackslash item}
   \index{item2@\textbackslash item{[}{]}}
\item[itemize] Items have no boldfaced description, instead every item
   gets a bullet.
   \index{begin{itemize}1@\textbackslash begin\{itemize\}}
   \index{begin{itemize}2@\textbackslash begin\{itemize*\}}
   \index{end{itemize}1@\textbackslash end\{itemize\}}
   \index{end{itemize}2@\textbackslash end\{itemize*\}}
\item[enumerate] Items have also no description, but are numbered
   \index{begin{enumerate}1@\textbackslash begin\{enumerate\}}
   \index{begin{enumerate}2@\textbackslash begin\{enumerate*\}}
   \index{end{enumerate}1@\textbackslash end\{enumerate\}}
   \index{end{enumerate}2@\textbackslash end\{enumerate*\}}
\end{description*}

\noindent
Above enumeration is made as follows:

\begin{lstlisting}[language=]
\begin{description*}
\item[description] This enumeration itself is of description type
\item[itemize] Items have no boldfaced description, instead every item
   gets a bullet.
\item[enumerate] Items have also no description, but are numbered
\end{description*}
\end{lstlisting}

\noindent
Every enumeration has two forms:
\begin{enumerate}
\item Without asterisk. \\
   Results in vertical spacing between the items. This enumeration itself is
   without asterisk.
\item With asterisk. \\
   Results in \emph{no} vertical spacing between the items. Former enumeration
   used such style.
\end{enumerate}

\noindent
This enumeration was made as follows:

\begin{lstlisting}[language=]
\begin{enumerate}
\item Without asterisk. \\
   Results in vertical spacing between the items. This enumeration itself is
   without asterisk.
\item With asterisk. \\
   Results in \emph{no} vertical spacing between the items. Former enumeration
   used such style.
\end{enumerate}
\end{lstlisting}

%------------------------------------------------------------------------------
\section{Referring, citing, quoting and indexing}\label{sec:LaTeX.Ref}
%------------------------------------------------------------------------------

Headings, figures, listings, etc. can be labeled with a tag. The section
"headings" is labeled e.g. as follows:

\index{label@\textbackslash label\{\}}
\begin{lstlisting}[language=]
%------------------------------------------------------------------------------
\section{Headings}\label{sec:LaTeX.Headings}
%------------------------------------------------------------------------------
\end{lstlisting}

\index{S@\textbackslash S}
\index{ref@\textbackslash ref\{\}}
\index{pageref@\textbackslash pageref\{\}}

\noindent
Now it is possible to refer to this section with e.g.\\
!\S\ref{sec:LaTeX.Headings} p\pageref{sec:LaTeX.Headings}! which will result in
\S\ref{sec:LaTeX.Headings} p\pageref{sec:LaTeX.Headings}.

In this project we have a label naming convention. Use the following prefixes:
\begin{description*}
\item[sec:] For chapters, sections and subsections
\item[fig:] For figures
\item[tab:] For tables
\item[lst:] For listings
\end{description*}

\index{cite@\textbackslash cite\{\}}

\noindent
You can refer to an external source with e.g. !\cite{LaTeX}! which will look
like \cite{LaTeX}. In the bibliography file !kubicas.lib! an item with the
label !LaTeX! exists already.

To include a quote in your text (just like the quotes from WikiPedia at the
beginning of this chapter) you can use:

\index{begin{quote}@\textbackslash begin\{quote\}}
\index{end{quote}@\textbackslash end\{quote\}}
\begin{lstlisting}[language=]
\begin{quote}
...
\end{quote}
\end{lstlisting}

\index{ix@\textbackslash ix\{\}}
\index{index@\textbackslash index\{\}}

\noindent
To mark a word in the text to be added to the index you can use !\ix{word}!.
Sometimes the word in the text has a different conjugation then the conjugation
you like to have in the index. Then you can use !form1\index{form2}!.
Finally, it is possible to influence the sort order in the index. e.g.
!\index{\textbackslash index\{\}}! will be placed under '\textbackslash', but
!\index{index@\textbackslash index\{\}}! can be found under the 'i'.

%------------------------------------------------------------------------------
\section{UML figures}
%------------------------------------------------------------------------------

LaTeX has many very different ways to incorporate a figure into the text. 

\noindent
The environment !figure! creates a floating block. LaTeX will calculate the
best position to embed the picture in the text. The environment !center! will
center the figure horizontally.
bla bla bla bla bla bla
The !\caption! will put some text underneath the
picture and give it a number. The !\label! is already discussed in
\S\ref{sec:LaTeX.Ref} p\pageref{sec:LaTeX.Ref}.

%------------------------------------------------------------------------------
\section{Source listings}
%------------------------------------------------------------------------------

It is not the intention to include long listings of source code in the
documentation. Still it is very common to show the principle of certain
constructions in a few lines. In this project we use two source listing styles:

\begin{enumerate*}
\item As a floating object, with border, label and caption.
\item Simple: just the listing, no label, no caption, no border, not floating
\end{enumerate*}

\noindent
Listing~\ref{lst:LaTeX.lst.float} p\pageref{lst:LaTeX.lst.float} is a floating
object and is made as follows:

\begin{lstlisting}[
   float,
   frame=trBL,
   caption={Listing as a floating object},
   label={lst:LaTeX.lst.float},
   gobble=3]
   for( int i(0); i<10; ++i )
   {
      // do something
   }
\end{lstlisting}

\index{begin{lstlisting}[]@\textbackslash begin\{lstlisting\}{[}{]}}
\index{end{lstlisting}@\textbackslash end\{lstlisting\}}
\index{gobble}
{\footnotesize
\begin{verbatim}
\begin{lstlisting}[
   float,
   frame=trBL,
   caption={Listing as a floating object},
   label={lst:LaTeX.lst.float},
   gobble=3]
   for( int i(0); i<10; ++i )
   {
      // do something
   }
\end{lstlisting}
\end{verbatim} }

\noindent
The default language is C++, and LaTeX will take care of syntax highlighting.
A special parameter to remember is !gobble!. With !gobble! you can define the
number of spaces that need to be ignored at the beginning of every line in the
listing.

The simple form of listing (which is intended to be used most often) looks as
follows:
\begin{lstlisting}
for( int i(0); i<10; ++i )
{
   // do something
}
\end{lstlisting}

\noindent
and is made with:
{\footnotesize
\begin{verbatim}
\begin{lstlisting}
for( int i(0); i<10; ++i )
{
   // do something
}
\end{lstlisting}
\end{verbatim} }

\noindent
When using identifiers of a listing in the text, it is appropriate to typeset
it in the same font as the listing. That can be done with
\lstinline[language=,basicstyle=\ttfamily]|!<some inline text>!| which results
in \\
!<some inline text>!.

%------------------------------------------------------------------------------
\section{No indention after vertical whitespace}
%------------------------------------------------------------------------------

LaTeX indents the first line when a new paragraph starts. But e.g. the first
paragraph in a section is not indented. It is considered to be aesthetic that a
paragraph is \emph{not} indented when it is preceded by vertical whitespace.
Unfortunatelly LaTeX doesn't do so automatically. So after e.g. listings and
enumerations we need to instruct LaTeX not to indent with the !\noindent!
command. \index{noindent@\textbackslash noindent}

%------------------------------------------------------------------------------
\section{Miscellaneous}
%------------------------------------------------------------------------------

A few miscellaneous LaTeX commands:

\begin{itemize}
\index{emph{}@\textbackslash emph\{\}}
\item !\emph{}! to emphasize a word like \emph{this}.
\index{tbc@\textbackslash tbc}
\item !\tbc! to indicate that text is unfinished: \tbc
\index{abbr{}@\textbackslash abbr\{\}}
\item !\abbr{}! for abbreviations like tbc = \abbr{t}o
      \abbr{b}e \abbr{c}ontinued, which is made with \\
      !tbc = \abbr{t}o \abbr{b}e \abbr{c}ontinued!
\index{begin{Note}@\textbackslash begin\{Note\}}
\index{end{Note}@\textbackslash end\{Note\}}
\item !\begin{Note} \end{Note}! for a labeled note
\index{begin{Definition}@\textbackslash begin\{Definition\}}
\index{end{Definition}@\textbackslash end\{Definition\}}
\item !\begin{Definition} \end{Definition}! for a labeled definition
\index{Rightarrow@\$\textbackslash Rightarrow\$}
\item !$\Rightarrow$! which results in: $\Rightarrow$
\index{textbackslash@\textbackslash textbackslash}
\item !\textbackslash! to print a backslash: \textbackslash
\index{nth{}@\textbackslash nth\{\}}
\item !\nth{}! for superscript counting numbers,
      like \nth{1}, \nth{2}, \nth{3}, etc.
\index{~@\textasciitilde}
\item !~! represents an unbreakable space, e.g. for
      !listing~\ref{lst:LaTeX.lst.float}!
\end{itemize}

%------------------------------------------------------------------------------
\section{Installation for Windows}
%------------------------------------------------------------------------------

To install LaTeX for as we use it in this project you need to install all
packages of the following sub sections.

%..............................................................................
\subsection{Install Gostscript}

\begin{enumerate*}
\item Navigate to the "LaTeX" directory
\item Right click !gs914w64.exe! and select "run as administrator"
\item Setup
\item Next
\item I Agree
\item Destination folder: !C:\Program Files\gs\gs9.14!
\item Install
\item On: "Generate cidfmap for Windows CJK TrueType fonts" for Chinese,
      Japanese and Korean fonts
\item Off: Show readme
\item Finish
\end{enumerate*}

%..............................................................................
\subsection{Install gsview}

\begin{enumerate*}
\item Navigate to the "LaTeX" directory
\item Right click !gsv50w64.exe! and select "run as administrator"
\item Setup
\item English
\item Next
\item Next
\item On: "Associate PostScript (.ps and .eps) files with GSview"
\item Off: "Associate PDF (.pdf) files with GSview"
\item Next
\item Destination folder: !C:\Program Files\Ghostgum!
\item Next
\item Next
\item Finish
\item Exit
\end{enumerate*}

%..............................................................................
\subsection{Install MikTex}

\begin{enumerate*}
\item Navigate to the "LaTeX" directory
\item Right click !basic-miktex-2.9.5105-x64.exe! and select "run as
      administrator"
\item Accept the MiKTeX copying conditions
\item Next
\item Anyone
\item Next
\item Destination folder: !C:\Program Files\MiKTeX 2.9!
\item Next
\item "A4" and "Ask me first"
\item Next
\item Start (takes 2 minutes)
\item Next
\item Close
\item Press the Windows start button
\item Select "All programs"
\item Select "MiKTeX 2.9"
\item Select "Maintenance (Admin)"
\item Select "Update (Admin)"
\item Next
\item Next
\item Next
\item Finish
\end{enumerate*}

\begin{Note}
On \url{http://www.miktex.org/download} a link is/was present to an
installation tutorial and to a "run the update wizard" tutorial
\end{Note}

%------------------------------------------------------------------------------
\section{Installation for Linux}
%------------------------------------------------------------------------------

\tbc

%------------------------------------------------------------------------------
\section{How LaTeX for Windows was downloaded}
%------------------------------------------------------------------------------

For the project, it is preferred that every participant uses the same version
of tools. This also applies for LaTeX. Therefore it is not preferred that
participants download their 'own' or the newest version of a tool. Instead an
installation CD (or an ISO image) is prepared, which contains all appropriate
versions. So please use the CD instead of following instructions below. The
instructions below are only included to be able to create a new version of the
installation CD.

%..............................................................................
\subsection{How Gostscript was downloaded}

\begin{itemize*}
\item Date of download: May \nth{2}, 2014
\end{itemize*}
\begin{enumerate*}
\item went to \url{http://www.ghostscript.com}
\item Under "Releases and News", click "here" of "Ghostscript releases can be
      downloaded here".
\item Click "Ghostscript 9.14"
\item Selected "Ghostscript GPL Release" of "Ghostscript 9.14 for Windows
      (64 bit)"
\item Selected "Save as", and navigated to the "Installation" directory
\item The file !gs914w64.exe! was downloaded
\end{enumerate*}

%..............................................................................
\subsection{How gsview was downloaded}

\begin{itemize*}
\item Date of download: May \nth{2}, 2014
\end{itemize*}
\begin{enumerate*}
\item went to \url{http://pages.cs.wisc.edu/~ghost/gsview/}
\item Under "GSview Software", select "GSview release v5.0"
\item Under "Obtaining GSview", select "gsv50w64.exe"
\item Selected "Save as", and navigated to the "Installation" directory
\item The file !gsv50w64.exe! was downloaded
\end{enumerate*}

%..............................................................................
\subsection{How MikTeX was downloaded}

\begin{itemize*}
\item Date of download: May \nth{2}, 2014
\end{itemize*}
\begin{enumerate*}
\item went to \url{http://www.miktex.org}
\item Selected "Download"
\item Selected "Other downloads"
\item Selected "Basic MiKTeX 2.9.5105 64-bit Installer, Version 2.9.5105
      Windows 64-bit, Size: 158.47 MB"
\item Selected "Save as", and navigated to the "Installation" directory
\item The file !basic-miktex-2.9.5105-x64.exe! was downloaded
\end{enumerate*}

%------------------------------------------------------------------------------
\section{How LaTeX for Linux was downloaded}
%------------------------------------------------------------------------------

\tbc
